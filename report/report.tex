\documentclass[a4paper,10pt]{article}

\usepackage{amsmath}
\usepackage[utf8]{inputenc}
\usepackage[left=2.5cm,right=2.5cm,top=2.5cm,bottom=2.0cm]{geometry}

\title{Modelling and Simulation of Systems\\ \Large
Exercise 3: Pseudorandom number generators}
\author{Agata Radys, Paweł Cejrowski, Łukasz Myśliński}
\date{\today}

\pdfinfo{%
  /Title    (Modelling and Simulation of Systems - Exercise 3: Pseudorandom number generators)
  /Author   (Agata Radys, Paweł Cejrowski, Łukasz Myśliński)
  /Creator  (Paweł Cejrowski)
  /Producer (Paweł Cejrowski)
  /Subject  (Modelling and Simulation of Systems)
  /Keywords (pseudorandom; generators; modelling and simulation of systems)
}

\begin{document}
\maketitle
\textbf{Tester:} Paweł Cejrowski



\section{Generator 1}

\subsection{Equation}
Numbers are generated according to equation \ref{eq:1}.
\begin{equation}
\label{eq:1}
  x_{n+1} = x_n \oplus x_{n-1}\ \ (mod\ 2^{32})
\end{equation}
where ${x_0}$ and ${x_1}$ are given.
\subsection{Questions}
\begin{itemize}
 \item \textbf{What is the minimum(maximum) possible value of the period? Give an example of initial values for which the period is small(large).} \\
Minimum value of period is equal $1$, when both $x_0$ and $x_1$ are equal $0$. Maximum value of period is $3$ when $x_0$ or $x_1$ is different from 0(e.g. $x_0 = 0$ and $x_1 = 16$).

 \item \textbf{What is the minimum(maximum) possible mean value? Give an example of initial values for which the average value is small(large).} \\
Minimum possible mean value is equal $0$, when both $x_0$ and $x_1$ are equal $0$. Maximum founded mean value of period is equal $\frac{2}{3} \cdot (2^{32} - 1) = 2863311530$, when $x_0 = x_1 = 2^{32} -1$

 \item \textbf{What is the minimum(maximum) possible variance? Give an example of initial values for which the variance is small(large).} \\
Minimum possible variance value is equal $0$, when both $x_0$ and $x_1$ are equal $0$. Maximum founded variance value of period is equal $4.0992764589155E+18$, when $x_0 = x_1 = 2^{32} -1$

 \item \textbf{Does the generator meet the requirements that good generators should satisfy? If not, which of the requirements are not satisfied and why?} \\
  The generator does \textbf{not} meet all requirements for good generators. \\
  Characteristics of good generators:
   \begin{enumerate}
    \item \textbf{generated numbers distributions are as close as possible to the desired one} \\
    Not satisfied: 3 point distribution.
    \item \textbf{subsequences of the produced sequence are mutually independent} \\
    Not satisfied: each subsequence of length 3 contains the same 3 numbers.
    \item \textbf{long period, with length at least $\sqrt{n}$, where $n$ is the length of the used subsequence} \\
    Not satisfied: short period with maximum length of 3.
    \item \textbf{the ability to make jumps, i.e. to compute $x_j$ from $x_i$ for every $j > i$} \\
    Not satisfied: $x_j$ cannot be computed from single number $x_i$, at least one other number $x_k$, where $k(mod\ 3) \neq i(mod\ 3)$  needed.
    \item \textbf{repeatable, portable and efficient} \\
    Satisfied(with minor objections to efficiency, due to inability of making jumps).
   \end{enumerate}
 
 \item \textbf{Is the generator suitable for use in the cryptography? If not, why?} \\
  The generator is \textbf{not} suitable for use in the cryptography, because it does not satisfy any of the desired conditions.\\
 Characteristics of generators suitable for use in the cryptography:
  \begin{enumerate}
   \item \textbf{it must be impossible to predict its seed and internal state even if we have a large sample of the numbers it produced} \\
   Not satisfied: it is very easy to predict the number when the period is at most 3 numbers long.
   \item \textbf{it must have a long period for every possible value of its seed} \\
   Not satisfied: do not have long period for any seed.
   \item \textbf{it should be unpredictable to the public, i.e. the probability of predicting the subsequent numbers should be low even if have a large sample of the numbers it produced} \\
   Not satisfied: having 3 consecutive numbers one can predict any other.
   \end{enumerate}

\end{itemize}

\section{Generator 2}
\subsection{Equation}
Numbers are generated according to equation \ref{eq:2}.
\begin{equation}
\label{eq:2}
  x_{n+1} = 3 \cdot x_n - 1\ \ (mod\ 2^{32}) 
\end{equation}
where ${x_0}$ is given.





\section{Generator 3}
\subsection{Equation}
Numbers are generated according to equation \ref{eq:3}.
\begin{equation}
\label{eq:3}
\begin{cases}
x'_{n+1} = x'_n \oplus x'_{n-1}\ \ (mod\ 2^{32}) \\ x''_{n+1} = 3 \cdot x_n'' - 1\ \ (mod\ 2^{32}) \\ x_{n+1} = x'_{n+1} \cdot x''_{n+1}\ \ (mod\ 2^{32})
\end{cases}
\end{equation}
where ${x'_0}$, ${x'_1}$ and ${x''_0}$ are given.
\end{document}

