\documentclass[a4paper,10pt]{article}
\usepackage{amsmath}
\usepackage{amssymb}
\usepackage{pifont}
\usepackage[utf8]{inputenc}
\usepackage[left=2.5cm,right=2.5cm,top=2.5cm,bottom=2.0cm]{geometry}
\usepackage{multirow}
\usepackage{geometry}
\usepackage{pdflscape}

\newcommand{\xmark}{\ding{55}}%


\title{Modelling and Simulation of Systems\\ \Large
Exercise 3: Pseudorandom number generators}
\author{Agata Radys, Paweł Cejrowski, Łukasz Myśliński}
\date{\today}

\pdfinfo{%
  /Title    (Modelling and Simulation of Systems - Exercise 3: Pseudorandom number generators)
  /Author   (Agata Radys, Paweł Cejrowski, Łukasz Myśliński)
  /Creator  (Paweł Cejrowski)
  /Producer (Paweł Cejrowski)
  /Subject  (Modelling and Simulation of Systems)
  /Keywords (pseudorandom; generators; modelling and simulation of systems)
}

\begin{document}
\newgeometry{margin=1.8cm}
\maketitle
\textbf{Tester:} Paweł Cejrowski



\section{Generator 1}

GFSR(generalized feedback shift register) with parameters: $j=1$, $k=2$, $m=2^{32}$.
\subsection{Equation}
Numbers are generated according to equation \ref{eq:1}.
\begin{equation}
\label{eq:1}
  x_n = x_{n-1} \oplus x_{n-2}\ \ (mod\ 2^{32})
\end{equation}
where ${x_0}$ and ${x_1}$ are given.


\subsection{Questions}
\begin{itemize}
 \item \textbf{What is the minimum(maximum) possible value of the period? Give an example of initial values for which the period is small(large).} \\
Minimum value of period is equal $1$, when both $x_0$ and $x_1$ are equal $0$. Based on lecture, we know that maximum value of period is
$( 2^k - 1 ) \cdot k = ( 2^2 - 1) \cdot 2 = 6$, but this value of period was not found during tests. Found $3$ when $x_0$ or $x_1$ was different from 0(e.g. $x_0 = 0$ and $x_1 = 16$).

 \item \textbf{What is the minimum(maximum) possible mean value? Give an example of initial values for which the average value is small(large).} \\
Minimum possible mean value is equal $0$, when both $x_0$ and $x_1$ are equal $0$. Maximum founded mean value within period is equal $\frac{2}{3} \cdot (2^{32} - 1) = 2863311530$, when $x_0 = x_1 = 2^{32} -1$

 \item \textbf{What is the minimum(maximum) possible variance? Give an example of initial values for which the variance is small(large).} \\
Minimum possible variance value is equal $0$, when both $x_0$ and $x_1$ are equal $0$. Maximum founded variance value within period is equal $4.0992764589155E+18$, when $x_0 = x_1 = 2^{32} -1$

 \item \textbf{Does the generator meet the requirements that good generators should satisfy? If not, which of the requirements are not satisfied and why?} \\
  The generator does \textbf{not} meet all requirements for good generators. \\
  Characteristics of good generators:
   \begin{enumerate}
    \item \textbf{generated numbers distributions are as close as possible to the desired one} \\
    Not satisfied: 3 point distribution.
    \item \textbf{subsequences of the produced sequence are mutually independent} \\
    Not satisfied: each subsequence of length 3 contains the same 3 numbers.
    \item \textbf{long period, with length at least $\sqrt{l}$, where $l$ is the length of the used subsequence} \\
    Not satisfied: short period with maximum length of 3.
    \item \textbf{the ability to make jumps, i.e. to compute $x_j$ from $x_i$ for every $j > i$} \\
    Not satisfied: $x_j$ cannot be computed from single number $x_i$, at least one other number $x_k$, where $k(mod\ 3) \neq i(mod\ 3)$  needed.
    \item \textbf{repeatable, portable and efficient} \\
    Satisfied.
   \end{enumerate}
 
 \item \textbf{Is the generator suitable for use in the cryptography? If not, why?} \\
  The generator is \textbf{not} suitable for use in the cryptography, because it does not satisfy any of the desired conditions.\\
 Characteristics of generators suitable for use in the cryptography:
  \begin{enumerate}
   \item \textbf{it must be impossible to predict its seed and internal state even if we have a large sample of the numbers it produced} \\
   Not satisfied: it is very easy to predict the number when the period is at most 3 numbers long.
   \item \textbf{it must have a long period for every possible value of its seed} \\
   Not satisfied: do not have long period for any seed.
   \item \textbf{it should be unpredictable to the public, i.e. the probability of predicting the subsequent numbers should be low even if have a large sample of the numbers it produced} \\
   Not satisfied: having 3 consecutive numbers one can predict any other.
   \end{enumerate}

\end{itemize}

\section{Generator 2}
LCG(linear congruential generator) with parameters $a=3$, $c=-1$, $m=2^{32}$.
\subsection{Equation}
Numbers are generated according to equation \ref{eq:2}.
\begin{equation}
\label{eq:2}
  x_n = 3 \cdot x_{n-1} - 1\ \ (mod\ 2^{32}) 
\end{equation}
where ${x_0}$ is given.

\subsection{Questions}
Below corner cases were found by $10000$ generator executions with randomized $n$, $m$ and $seed$ by shell \texttt{\$RANDOM}, which generates pseudorandom number from range $0$-$32767$.
\begin{itemize}
 \item \textbf{What is the minimum(maximum) possible value of the period? Give an example of initial values for which the period is small(large).} \\ 
Minimum value of period is equal $m-n + 1$ for $m<100000$.  Based on lecture, we know that maximum value of period is $2^30$ and is reached when $x_0$ is odd.

 \item \textbf{What is the minimum(maximum) possible mean value? Give an example of initial values for which the average value is small(large).} \\
Minimum mean value for longer ranges($m-n>50$) is bigger than $1e9$. Maximum founded mean value is $2373236302.0435$, but still it is only empirical.

 \item \textbf{What is the minimum(maximum) possible variance? Give an example of initial values for which the variance is small(large).} \\
Minimum and maximum variance value for longer ranges($m-n>50$) is bigger than $1e18$.

 \item \textbf{Does the generator meet the requirements that good generators should satisfy? If not, which of the requirements are not satisfied and why?} \\
  The generator meets all requirements for good generators. \\
  Characteristics of good generators:
   \begin{enumerate}
    \item \textbf{generated numbers distributions are as close as possible to the desired one} \\
    Satisfied: $K^+$ and $K^-$ satisfied in $99.9\%$ cases with the $\alpha = 0.05$. All satisfied with $\alpha = 0.15$.
    \item \textbf{subsequences of the produced sequence are mutually independent} \\
    Satisfied: chi-square always satisfied on the $\alpha = 0.05$.
    \item \textbf{long period, with length at least $\sqrt{l}$, where $l$ is the length of the used subsequence} \\
    Satisfied: proportional to $l$.
    \item \textbf{the ability to make jumps, i.e. to compute $x_j$ from $x_i$ for every $j > i$} \\
    Satisfied.
    \item \textbf{repeatable, portable and efficient} \\
    Satisfied.
   \end{enumerate}
 
 \item \textbf{Is the generator suitable for use in the cryptography? If not, why?} \\
  The generator is \textbf{not} suitable for use in the cryptography, because it does not satisfy two of the desired conditions.\\
 Characteristics of generators suitable for use in the cryptography:
  \begin{enumerate}
   \item \textbf{it must be impossible to predict its seed and internal state even if we have a large sample of the numbers it produced} \\
   Not satisfied: it is possible to predict seed internal state based on subsequence.
   \item \textbf{it must have a long period for every possible value of its seed} \\
   Satisfied.
   \item \textbf{it should be unpredictable to the public, i.e. the probability of predicting the subsequent numbers should be low even if have a large sample of the numbers it produced} \\
   Not satisfied: the formula is easy to guess based on consecutive numbers.
   \end{enumerate}

\end{itemize}


\section{Generator 3}
\subsection{Equation}
Numbers are generated according to equation \ref{eq:3}.
\begin{equation}
\label{eq:3}
\begin{cases}
x'_n = x'_{n-1} \oplus x'_{n-2}\ \ (mod\ 2^{32}) \\ x''_n = 3 \cdot x_{n-1}'' - 1\ \ (mod\ 2^{32}) \\ x_n = x'_n \cdot x''_n\ \ (mod\ 2^{32})
\end{cases}
\end{equation}
where ${x'_0}$, ${x'_1}$ and ${x''_0}$ are given.

\subsection{Questions}
Below corner cases were found by $1000$ generator executions with randomized $n$, $m$ and seeds ${x'_0}$, ${x'_1}$ and ${x''_0}$ by shell \texttt{\$RANDOM}, which generates pseudorandom number from range $0$-$32767$.
\begin{itemize}
 \item \textbf{What is the minimum(maximum) possible value of the period? Give an example of initial values for which the period is small(large).} \\ 
Value of period is proportional to $m-n + 1$ for $m-n<100000$.

 \item \textbf{What is the minimum(maximum) possible mean value? Give an example of initial values for which the average value is small(large).} \\
Minimum and maximum values were over $2000000$ in tests.
 
 \item \textbf{What is the minimum(maximum) possible variance? Give an example of initial values for which the variance is small(large).} \\
Minimum and maximum values were over $1.7e18$ in tests.

 \item \textbf{Does the generator meet the requirements that good generators should satisfy? If not, which of the requirements are not satisfied and why?} \\
  The generator does \textbf{not} meet all requirements for good generators. \\
  Characteristics of good generators:
   \begin{enumerate}
    \item \textbf{generated numbers distributions are as close as possible to the desired one} \\
    Satisfied: $K^+$ and $K^-$ satisfied in $99.9\%$ cases with the $\alpha = 0.011$. All satisfied with $\alpha = 0.10$.
    \item \textbf{subsequences of the produced sequence are mutually independent} \\
    Satisfied: chi-square always satisfied on the $\alpha = 0.05$.
    \item \textbf{long period, with length at least $\sqrt{l}$, where $l$ is the length of the used subsequence} \\
    Satisfied: proportional to $l$.
    \item \textbf{the ability to make jumps, i.e. to compute $x_j$ from $x_i$ for every $j > i$} \\
    Not satisfied: $x_{i-1}$ also needed.
    \item \textbf{repeatable, portable and efficient} \\
    Satisfied.
   \end{enumerate}
 
 \item \textbf{Is the generator suitable for use in the cryptography? If not, why?} \\
  The generator is suitable for use in the cryptography.\\
 Characteristics of generators suitable for use in the cryptography:
  \begin{enumerate}
   \item \textbf{it must be impossible to predict its seed and internal state even if we have a large sample of the numbers it produced} \\
   Satisfied: strength is based on modulo factorization.
   \item \textbf{it must have a long period for every possible value of its seed} \\
   Satisfied.
   \item \textbf{it should be unpredictable to the public, i.e. the probability of predicting the subsequent numbers should be low even if have a large sample of the numbers it produced} \\
   Satisfied: strength is based on modulo factorization.
   \end{enumerate}

\end{itemize}




\newgeometry{margin=1cm}
\begin{landscape}
\appendix
\section{Statistics comparisons}
\begin{table}[ht!]
\centering
\caption{Comparisons of minimum and maximum values of period, mean and variance}
\label{tab:comparisons}
  \begin{tabular}{cc|c|l|c|l|c|l|}
  \cline{3-8}
						  &     & \multicolumn{2}{|c|}{Generator 1} & \multicolumn{2}{|c|}{Generator 2} & \multicolumn{2}{|c|}{Generator 3} \\ \hline
  \multicolumn{2}{|c|}{Equation}                        & \multicolumn{2}{|c|}{$x_n = x_{n-1} \oplus x_{n-2}\ \ (mod\ 2^{32})$} & \multicolumn{2}{|c|}{$y_n = 3 \cdot y_{n-1} - 1\ \ (mod\ 2^{32})$} & \multicolumn{2}{|c|}{$z_n = x_n \cdot y_n\ \ (mod\ 2^{32})$} \\ \hline
  \multicolumn{2}{|c|}{}                                & value & condition & value & condition & value & condition \\ \hline \hline
  \multicolumn{1}{|c|}{\multirow{2}{*}{period}}   & min & $1$ & $x_0=x_1=0$                               &  \multicolumn{2}{|c|}{proportional to $m-n$(tested up to 100000)} & \multicolumn{2}{|c|}{proportional to $m-n$(tested up to 100000)}  \\ \cline{2-8} 
  \multicolumn{1}{|c|}{}                          & max & $3$ & elsewhere                                 & $2^{30}$ & $x_0$ is odd      & \multicolumn{2}{|c|}{proportional to $m-n$(tested up to 100000)}  \\ \hline
  \multicolumn{1}{|c|}{\multirow{2}{*}{mean}}     & min & $0$ & $x_0=x_1=0$                               & \multicolumn{2}{|c|}{\xmark} & \multicolumn{2}{|c|}{\xmark} \\ \cline{2-8} 
  \multicolumn{1}{|c|}{}                          & max & $2863311530$ & $x_0=x_1=2^{32}-1$               & \multicolumn{2}{|c|}{\xmark} & \multicolumn{2}{|c|}{\xmark} \\ \hline
  \multicolumn{1}{|c|}{\multirow{2}{*}{variance}} & min & $0$ & $x_0=x_1=0$                               & \multicolumn{2}{|c|}{\xmark} & \multicolumn{2}{|c|}{\xmark} \\ \cline{2-8} 
  \multicolumn{1}{|c|}{}                          & max & *$4.0992764589155E+18$ & $x_0=x_1=2^{32}-1$     & \multicolumn{2}{|c|}{\xmark} & \multicolumn{2}{|c|}{\xmark} \\ \hline
  \end{tabular}
\end{table}
* - founded during tests(not analytically checked)
\section{Good generator requirements}
\begin{table}[ht!]
\centering
\caption{Comparison of meeting requirements for good generators}
\label{tab:goodgenerator}
\begin{tabular}{l|c|c|c|}
  \cline{2-4}
                                                                                              & Generator 1 & Generator 2 & Generator 3 \\ \hline
\multicolumn{1}{|p{14cm}|}{generated numbers distributions are as close as possible to the desired one}                   & \xmark      & \checkmark            &        \checkmark     \\ \hline
\multicolumn{1}{|p{14cm}|}{subsequences of the produced sequence are mutually independent}                                & \xmark      & \checkmark            &        \checkmark     \\ \hline
\multicolumn{1}{|p{14cm}|}{long period, with length at least $\sqrt{n}$, where $n$ is the length of the used subsequence} & \xmark      & \checkmark            &        \checkmark     \\ \hline
\multicolumn{1}{|p{14cm}|}{the ability to make jumps, i.e. to compute $x_j$ from $x_i$ for every $j > i$}                 & \xmark      & \checkmark            &        \xmark     \\ \hline
\multicolumn{1}{|p{14cm}|}{repeatable, portable and efficient}                                                            & \checkmark  & \checkmark            &        \checkmark     \\ \hline
\end{tabular}
\end{table}

\section{Cryptography suitability}
\begin{table}[ht!]
\centering
\caption{Comparison of cryptography suitability}
\label{tab:cryptography}
\begin{tabular}{c|c|c|c|}
  \cline{2-4}
                                                                                              & Generator 1 & Generator 2 & Generator 3 \\ \hline
  \multicolumn{1}{|p{14cm}|}{it must be impossible to predict its seed and internal state even if we have a large sample of the numbers it produced } & \xmark      &     \xmark          &       \checkmark      \\ \hline
  \multicolumn{1}{|p{14cm}|}{it must have a long period for every possible value of its seed}                                & \xmark      &     \checkmark        &      \checkmark       \\ \hline
  \multicolumn{1}{|p{14cm}|}{it should be unpredictable to the public, i.e. the probability of predicting the subsequent numbers should be low even if have a large sample of the numbers it produced} & \xmark      &        \xmark       &     \checkmark        \\ \hline
\end{tabular}
\end{table}
   
\end{landscape}
\restoregeometry


\end{document}

